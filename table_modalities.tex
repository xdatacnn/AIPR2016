%Table with all the image modalities discussed in this paper

% Please add the following required packages to your document preamble:
% \usepackage[table,xcdraw]{xcolor}
% If you use beamer only pass "xcolor=table" option, i.e. \documentclass[xcolor=table]{beamer}
\begin{table*}[!]
\centering
\caption{Microscopic images of materials across scales: specifications and methods}
\label{table1}
\begin{tabular}{p{2cm}p{1.6cm}p{1.6cm}p{3cm}p{7.5cm}}
\hline
\rowcolor[HTML]{CCE5FF}
Materials  &  Resolution ($\mu$m)  &  Image \newline modality  &  Imaging contrast \newline mechanism  &  Data analysis for specimen quantification                                       \\
\hline
\rowcolor[HTML]{FFFFFF}
ceramic \newline composites & 0.65 to 1.3                               & microCT                               & X-ray attenuation contrast                    & detection of fibers, fiber breaks and cracks using graph-based and template matching ML algorithms. Sec.~\ref{subsec:cmc}. Fig.~\ref{fig:microct}.
\\
\hline
\rowcolor[HTML]{F6F6F6}
geological samples & 0.65 to 2.5  & microCT                               & X-ray attenuation contrast              & segregation of components from multiphase specimens using Markov Random Field ML algorithms. Sec.~\ref{subsec:cmc}. Fig.~\ref{fig:pmrf}.
\\
\hline
\rowcolor[HTML]{FFFFFF}
nanoparticle clusters & 0.2457 & SEM                               &
electron scattering                    & counting, topographical characterization, morphology, particle distribution, ensemble representativeness. Sec.~\ref{subsec:sem}. Fig.~\ref{fig:nanoparticles}.
\\
\hline
\rowcolor[HTML]{F6F6F6}
thin films   & 0.00164  & STEM \newline tomography                       & electron transmission                & pore organization across film, associated to level of pore coalescence; surface density analysis correlated to dielectric constant measurements. Sec.~\ref{subsec:stem}. Fig.~\ref{fig:stem}.
\\
\hline
\end{tabular}
\end{table*}
