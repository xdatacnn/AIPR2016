%Table with all the image modalities discussed in this paper

% Please add the following required packages to your document preamble:
% \usepackage[table,xcdraw]{xcolor}
% If you use beamer only pass "xcolor=table" option, i.e. \documentclass[xcolor=table]{beamer}
\begin{table*}[h!]
\centering
\caption{Scientific data under scrutiny with CNN: specifications and methods}
\label{table1}
\begin{tabular}{p{2cm}p{1.6cm}p{1.6cm}p{3cm}p{7.5cm}}
\hline
\rowcolor[HTML]{CCE5FF}
Specimen  &  Scale range  &  Image \newline modality  &  Imaging  \newline mechanism  &  Data analysis for quantitative microscopy
\\
\hline
\rowcolor[HTML]{FFFFFF} %chao
TFIIDM & $\mu$m & cryo-EM & electron scattering and transfer & Inspection of structural biology through macro-molecule projections using MatConvNet. Sec.~\ref{subsec:cryo}. Fig.~\ref{fig:cryo1}.
\\
\hline
\rowcolor[HTML]{F6F6F4} %chao
bragg peaks of what? & 0.001 to 0.009?  & crystallography? & X-ray diffraction  & Detection of Bragg peaks from whatever specimens using MatConvNet. Sec.~\ref{subsec:diffraction}. Fig.~\ref{fig:diffraction}.
\\
\hline
\rowcolor[HTML]{FFFFFF} %venkat
Thin films   & 0.00164?  & GISAXS  & X-ray scattering & Classification of crystal lattice structure using MatConvNet and Caffe as in Sec.~\ref{subsec:gisaxs}. Fig.~\ref{fig:gisaxs}.
\\
\hline
\rowcolor[HTML]{F6F6F4} %dani
Fiber beds & 0.65 to 1.3 & microCT  & X-ray attenuation contrast & detection of fiber profile using TensorFlow. Sec.~\ref{subsec:microct}. Fig.~\ref{fig:microct}.
\\
\hline
\end{tabular}
\end{table*}
