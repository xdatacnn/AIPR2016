%Table with all the image modalities discussed in this paper

% Please add the following required packages to your document preamble:
% \usepackage[table,xcdraw]{xcolor}
% If you use beamer only pass "xcolor=table" option, i.e. \documentclass[xcolor=table]{beamer}
\begin{table*}[!]
\centering
\caption{Scientific data under scrutiny with CNN: specifications and methods}
\label{table1}
\begin{tabular}{p{2cm}p{1.6cm}p{1.6cm}p{3cm}p{7.5cm}}
\hline
\rowcolor[HTML]{CCE5FF}
Materials  &  Resolution ($\mu$m)  &  Image \newline modality  &  Imaging  \newline mechanism  &  Data analysis for specimen quantification
\\
\hline
\rowcolor[HTML]{FFFFFF}
material2 \newline whatever & 0.65 to 1.3 & microCT                               & cryo-EM                     & detection something with MatConvNet. Sec.~\ref{subsec:cmc}. Fig.~\ref{fig:microct}.
\\
\hline
\rowcolor[HTML]{F6F6F6}
bragg peaks of what? & 0.65 to 2.5  & diffraction pattern & X-ray diffraction  & detection of Bragg peaks from whatever specimens using MatConvNet. Sec.~\ref{subsec:cmc}. Fig.~\ref{fig:pmrf}.
\\
\hline
\rowcolor[HTML]{FFFFFF}
material2 \newline whatever & 0.65 to 1.3 & microCT  & X-ray attenuation contrast & detection of fiber profile, TensorFlow. Sec.~\ref{subsec:cmc}. Fig.~\ref{fig:microct}.
\\
\hline
\rowcolor[HTML]{F6F6F6}
thin films   & 0.00164  & STEM \newline tomography  & electron transmission    & pore organization across film, associated to level of pore coalescence; surface density analysis correlated to dielectric constant measurements. Sec.~\ref{subsec:stem}. Fig.~\ref{fig:stem}.
\\
\hline
\end{tabular}
\end{table*}
